\documentclass[12pt]{article}
\usepackage{amsmath}
\usepackage{euler}
\usepackage{authblk}
\usepackage{tikz}
\usetikzlibrary{positioning}

\usepackage{hyperref}
\hypersetup{
    colorlinks,
    citecolor=black,
    filecolor=black,
    linkcolor=black,
    urlcolor=black
}

\usepackage{bookmark}
\usepackage{cite}
\usepackage{fancyhdr}

\usepackage[margin=1.13in]{geometry}

\pagestyle{fancy}
\fancyhf{}
\rhead{\rightmark}

\usepackage{titlesec}
\titleformat{\chapter}[display]
  {\normalfont\bfseries}{}{0pt}{\Huge}
\author{Junlong Gao}
\affil{}
\date{}

\newcommand{\Lconsistent}{linearizability }
\newcommand{\Sconsistent}{serializability }
\newcommand{\LConsistent}{Linearizability }
\newcommand{\SConsistent}{Serializability }

\newcommand{\wip}[1]{\textcolor{red}{[\textit{TODO: #1}]}}

\begin{document}
\tableofcontents

\title{A Survey on General and Special Purpose Filesystems}
\maketitle

The concept, design, implementation, and use cases of file systems has long
been challenged by a more structured approach of data like SQL databases and
key-value stores. Now, with the rise of object (blob) storage and block
storage (virtual SAN), its use cases are further being reconsidered and new
variants of file systems are implemented for specific tasks. This article
explores various modern filesystems and their similar storage system solutions, both in
terms of their design considerations, implementation variations, and the use
cases they are targeting for.

\section{Filesystems $=$ Namespaces Management $+$ Data Management}
What exactly are filesystems? Their essence are brought out only after their
replacements and variants are implemented. Consider Haystack, an implementation
for photo storage at Facebook~\cite{beaver2010finding}. They moved from their
initial NFS based implementation to their home-brew filesystem-like storage
system, namely Haystack, only to eliminate the namespace management provided by
the backing filesystems in NFS. If we look at a more traditional
general-purpose file system like Ceph, their paper cautiously admits its data
allocation scheme in fact reduced much of the system
complexity~\cite{weil2006ceph}.  Yet, their metadata system design (namespace
and its operations) are still subject to the complexity of filesystem tree
sharding/partitioning (scaling the namespace).

A tree-like (DAG, considering hardlinks) namespace structure, and various
operations on them like atomic, cyclic-free renames are challenges in its own
right for implementing a scalable, robust namespace storage. Failure modes are
more complicated, and it is harder to implement concurrency. Once this is
solved, data management itself is also challenging as files are essentially a
map from logical offset to data. Fragmentation, small IO, concurrency and
consistency in concurrent read/write access, all brings even the most
rudimentary design to its knees. To be POSIX-like filesystems with large scale,
most of the
implementations~\cite{ghemawat2003google,shvachko2010hadoop,weil2006ceph}
opts-in comprises like reduced POSIX guarantees, and large units of data
allocation to reduce mapping tables, or even no allocation table at
all~\cite{weil2006crush}.

These trade-offs are made to pave ways for scalability, yet harder problems
are yet to be solved. Both namespace and data have to be protected against
failure, and implementations of file systems are usually interleaved with
failure detection and data redundancy. In GFS and HDFS, writes are required
to participate in the replication protocol and have to deal with failures.
Ceph took a similar approach yet provides a way to acknowledge writes waiting
for propagation of all of the replication. The difference is how Ceph
explicitly took the system into 2 layers: the upper layer is the namespace
management and the lower layer is the object store (Reliable Autonomic
Distributed Object Store) which is built on a collection of plain files. In
this way, Ceph could worry about the scalability of the data and metadata
(namespace) separately, and build reliability in each layer (RADOS does the
underlying replication for data).

Yet with the rise of the object store, it seems apparent that the complicated
tree-like namespace hierarchy in a distributed storage system is more of a
burden than a help. Where are the namespaces? Photos names or video names are
structured inside a different store: the databases. They are not filenames
within a folder, rather are column fields in a SQL schema. Along with other
application-specific metadata, names are no longer part of the data storage.

Sequential read and writes are not only best for throughput, but also an
advantage in storage wearing, metadata management (structure sizing and
caching), and reducing fragmentation. Workloads in bulks are always better
than random, small frequent operations. Applications like databases usually
uses various techniques like write-ahead-logging with \texttt{fsync(2)} and
preallocation (\texttt{fallocate(1)}) for its storage access: it somehow has
its own management of data, and only needs a collection of files for
namespace purposes. On the other hand, thin provisioning, or sparse files,
provides a great level of flexibility as provisioning total space usage before
hand is hard and sometimes expensive. A good design should trade-off between
the granularity of metadata and the flexibility of thin provisioning.

\section{Space Allocations}
The ideal workload for a file system in a large scale, distributed manner, is 
to have IO access with good space locality (sequential, or in batch). Any such
file system implementing data management will have metadata to find where is the
actual physical storage of that piece of data. There are 3 questions around the
organization of this metadata:
\begin{itemize}
    \item 1) What is the space granularity in terms of allocation?
    \item 2) What is the write pattern for new spaces and overwrites?
    \item 3) How will space get freed?
\end{itemize}

In short, the discussion of space allocation is the discussion of data
lifecycle, which is an integral part of space management.

\begin{itemize}
    \item Let's start with the classic Google File System. It is more of a
    special-purpose filesystem than a general-purpose one. Its assumption is
    large, batched and sequential append, and large streaming
    reads~\cite{ghemawat2003google}. In this light, space granularity is
    generously large static-sized 64MB chunks. New space is resulted in
    append records to the file with aggressive client-side batching and
    retry. This also makes space management easy, as chunk size is both large
    and fixed, resulting in a simple table-based approach, mapping file
    offsets to chunk locators.

    \item A modern design like PolarFS~\cite{cao2018polarfs} takes the idea in
    GFS and advances it further. Data are stored in the unit of chunks and chunks
    are of 10GB sizing. A filesystem volume can contain multiple chunks for scaling
    data, and each chunk has a super block and a space allocation table, which maps:
    \[
        \texttt{chunk block offset} \rightarrow (\texttt{file inode}, \texttt{file offset})
    \]
    This allows each chunk to individually handle space allocation, and
    effectively shards the entire data management acrosses chunks. Of course,
    random writes and unmap will also be horrible in this schema. This
    mapping table size is governed by the physical granularity: 64KB. So a
    10GB chunk can have 128k entries at most, which fits in memory
    comfortably. Yet PolarFS is designed to support MySQL database workload,
    and database applications will always preallocate files and journals and
    effectively manage their data in these large files on their own.

    \item Ceph intentionally divides the whole filesystem into two layers: the
    metadata store MDS and the underlying storage devices ODS (object device
    store). ODS is implemented as RADOS (Reliable Autonomic Distributed
    Object Store), which handles most of the data allocation requests.
    Clients buffer write operations aggressively (using a form of lease
    called ``Capability Bits''~\cite{weil2006ceph}) and generate large-sized
    of objects. The file layout is from file offset to object ID. Finally, a
    sequence of placement and replication layer will assign an ODS node in
    RADOS for storing that object. Then we have 2 types of mappings, first is
    handled by MDS layer (\texttt{file offset} to \texttt{object ID}), and
    the second, \texttt{object ID} to \texttt{OSD offset}, is handled in the
    RADOS layer, both allocation and locating. The paper went on describing
    EBOFS for implementing each OSD, which is a plain large file in nodes'
    local EBOFS. I assume it has a table or tree-ish structure per file to
    translate and allocate objects.

    \item Finally, consider HayStack, evolved from load-balanced, NAS-based
    NFS~\cite{beaver2010finding}. The key idea in HayStack is it is an object
    store with a relatively flat namespace (since, structure information of
    the photos are captured in a different database), writes are always new
    writes, no overwrite of existing data but lots of random reads, and
    finally, there is no concept of file offset: a file is a photo and its
    data in entirety. The difficulties of the original NFS solution is the
    ``dilemma of the right RAM-to-disk ratio''. Still, all metadata must be
    able to reside in memory for serving random, long-tailed read requests.
    Data are appended in a store and a store can be thought as a large file
    with some index structure in memory for management of data. Allocation of
    data in the store is a flat table small enough to reside in memory, and
    new write always appends, both data and metadata. Read thus can be served
    with 1 disk IO using the in-memory table to find the offset in the store.
    Finally, such an append-only structure requires compaction for cleaning
    up deleted files. This is effectively a log-structured
    filesystem~\cite{rosenblum1992design}.
\end{itemize}

\section{IO Consistency and Metadata Sychonization}
The file-based abstraction provided by a filesystem is also responsible for
implementing sharing across different users. The other aspect, durability,
can also be thought as a form of sharing: sharing with itself across
different time points, while concurrent access is sharing with others at the
same time. Any form of sharing effectively implements a communication
channel. Indeed, local filesystems in UNIX-like kernels are heavily used as a
form of IPC (inter-process communication).

The semantics of concurrent read-write to a file defines what it means to be
shared, yet the most straightforward, layman's definition, is ``reads must
reflect the latest write''. On the other hand, sometimes the workload does
not have this use-case: it does not care what happens the same file is
shared, as the workload is inherently conflict-free (used as a
producer-consumer queue like GFS), or exclusive access (photos in Haystack
cannot be read after written, and immutable after written), or single process
access with its own concurrency control (MySQL on top of PolarFS).
Nevertheless, it is still worthwhile to discuss what are the consistency
levels and how they are implemented. Note in a distributed file system with
standard failure model, concurrency control of file access with replication
is as hard as distributed locking and subjects to CAP
theorem~\cite{gilbert2002brewer}.

\begin{itemize}
    \item Ceph implements a form of range locking over objects and use a
    lease-like approach for read-write synchronization. By default clients
    are encouraged to buffer writes and cache reads aggressively, using the
    ``Capability Bits'' provided from MDS as a form of lease. The consistency
    issue of these caching is solved by revoking the read/write permission in
    the bits when the MDS detects a mixed reader-writers have opened the same
    file. Writes to objects are controlledd by object locks, and buffering
    and caching will turn into synchronized operations after lease is
    revoked. Since Ceph is a general-purpose filesystem, it must favor these
    consistency guarantees in order to not break POSIX semantics. On the
    other hand, it does support \texttt{O\char`_LAZY} flag with syscalls to
    manually trigger caching and buffering, if applications neeed.

    \item GFS has to cope with its replication protocol. In GFS, the
    replication protocol of chunks (chunkserver redundancy) is explicit in
    the datapath, unlike PolarFS or Ceph. This opens a range of challenges,
    as clients have to deal with failed replications (half write and resync),
    in-order delivery (data can be applied, like append record, in different
    order across the replicas), duplication of data (retries) and placement
    decisions (should I ask for a new replica from the name node or should I
    wait?). Yet, writes are linearizable in the same
    chunk when there is no failure. This is achieved via lease-based locking:
    a single chunk server will be picked as the primary, and the client will
    push data to all replicas before talking to the primary. This push will
    not result in written-data being visible until the client commits it in
    the primary node. The primary node gathers all the different clients'
    mutation requests, assign them serial numbers, and ask the replicas for
    each request to apply them in the serial order, hence achieves
    linearizable reads. However, when write requests (offset $+$ size) are
    across multiple chunk boundaries, this can result in non-atomic write or
    interleaved with other append requests. Record append can be done in an
    atomic manner, but limited to the chunk size. This is less ideal compared
    to the modern replicated state machine approach.

    \item PolarFS uses a journal file (implemented by the Disk Paxos
    protocol~\cite{gafni2003disk}) to serialize metadata mutation (like
    punching holes in file or preallocation) and implement coordination. When
    node's journal anchor changes, writes will be able to refresh its state
    of the metadata. These only serializes ``new writes'', majority of which
    belong to MySQL preallocated database files. Readers does not need this
    synchronization, and readers are usually slave instances of MySQL
    databases for scaling reads.
\end{itemize}
\section{Layered Design and Data Placement}
Subsequent works after GFS takes the approach of separation of concerns.
PolarFS and Ceph intentionally separates the data path and the replication
protocol. The lower layer builds the abstraction of large, chunk-based storage
service. This service is fault tolerant in the sense that it implements its own
redundancy, without the upper layer's coordination.  Upper layer can focus on
data placement, sharding, metadata scale-out or synchronization, and leave the
complicated consistency issue of data replication to the lower layer.

\begin{itemize}
    \item Ceph uses its home-brew replication protocol. Data writes are
        journaled first in RADOS, and OSD will sync with each other on failure
        recovery. The cluster state is updated via heartbeats from OSD
        self-report, and controls data placement in terms of object ID. The
        most significant difference between Ceph and other distributed system
        is placement is not persisted, yet calculated by data signature and
        cluster state. There is no centralized service for looking up where is
        the data, yet still scale-out its capacity. I have read the failure
        detection section and recovery section closely, but I still believe it
        is less ideal than a more established approaching using replicated logs
        and a centralized service.

    \item PolarFS does the placement and replication in the lower layer. Data
        writes into chunk are replicated using its own proprietary
        \textit{ParallelRaft} protocol. Placement is in the unit of chunks, and
        it has its control plane for the chunk layer to dynamically move chunks
        around to mitigate hot spot issues.

    \item GFS does not have this layered approach. The datapath is fully aware
        the replication complexity and has to ask for replica and placement
        explicitly. GFS also does explicit chunk server movement and increase
        the number of replica to handle hot spot issues, and implements client
        side exponential random backoff to mitigate the thundering herd
        problem. All the placement and replication decisions can be done in the
        name node, which collects chunkserver heartbeats and scans for failures
        or latency/capacity issues.

    \item HayStack has its own directory service for placement decisions, and
        uses RAID for storage fault-tolerance. Proactive health check for
        storage layer access (\textit{Pitch-Fork}~\cite{beaver2010finding}).
\end{itemize}

\section{Namespace Operations}
In work.

\bibliography{main}{}
\bibliographystyle{plain}
\end{document}
