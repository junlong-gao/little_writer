\documentclass[12pt]{article}
\usepackage{amsmath}
\usepackage{euler}
\usepackage{authblk}
\usepackage{tikz}
\usetikzlibrary{positioning}

\usepackage{hyperref}
\hypersetup{
    colorlinks,
    citecolor=black,
    filecolor=black,
    linkcolor=black,
    urlcolor=black
}

\usepackage{bookmark}
\usepackage{cite}
\usepackage{fancyhdr}

\usepackage[margin=1.13in]{geometry}

\pagestyle{fancy}
\fancyhf{}
\rhead{\rightmark}

\usepackage{titlesec}
\titleformat{\chapter}[display]
  {\normalfont\bfseries}{}{0pt}{\Huge}
\author{Junlong Gao}
\affil{}
\date{}

\newcommand{\Lconsistent}{linearizability }
\newcommand{\Sconsistent}{serializability }
\newcommand{\LConsistent}{Linearizability }
\newcommand{\SConsistent}{Serializability }

\newcommand{\wip}[1]{\textcolor{red}{[\textit{TODO: #1}]}}

\begin{document}
\tableofcontents

\title{Spanner Transaction and Replication}
\maketitle

The Google's paper on building a ``globally consistent'' database
is a dense one encapsulating multiple interesting ideas. It uses locking
and two phase commit protocol to ensure \Sconsistent but the most
important, and significant contribution of this paper is how a assumed
``absolute time ordering'' can be used to achieve \Lconsistent~\cite{Cooper_2013}.

This document tries to highlight the ideas described in the paper,
with a focus on how the entire system achieves the external consistency
guarantee: {\Sconsistent} with {\Lconsistent}.

\section{The Architecture}
\subsection{Layered Structure}
Logically speaking, Spanner has a very clear layered structure:
\begin{itemize}
   \item The lowest layer is the persistent storage. Important design decision
   revolves around if it is supposed to be a shared distributed storage or not.
   Google uses Colossus (the successor of Google File System), a distributed
   file system.

   \item Then is the tablet, a local key-value store. One important aspect is the
    keys are versioned:
    \[
       (key:string, version:timestamp) \rightarrow val: string
    \]

    \item Then is the replication layer. Spanner uses Paxos to replicate and coordinate access of the key-value pairs in the tablets. The nodes
    responsible for replicating a particular set of key-value pairs is
    called a Paxos group, or a group, for short.

    \item Then is the transaction layer. Not every node is responsible for
    the transaction implementation at all times. Only the leader in a
    replication group needs to implement transaction support for the data in
    that group. This implies maintaining the lock table, participate in a
    transaction two-phase commit, recording prepare and commit messages.

    \item Then is the distribution layer. In Spanner, nodes are placed in
    different zones and there is a service \textit{zonemaster} for tracking nodes in each zone. Another \textit{placement driver} is responsible for
    tracking and balancing across zones.

    \item Finally, the entire deployment, including all the zones and the
    placement driver is named a single \textit{universe}. A \textit{universemaster} service is responsible for providing monitoring and
    interactive management for the entire deployment.
\end{itemize}

\subsection{Directories and Data Model}
A \textit{Directory} is a range of key-value pairs, sharing some prefix. This
is the smallest unit for replication, data re-balancing, and have application
level configurable replication factor. One Paxos group can replicate multiple
directories (since one node is one replica, note in the paper Google gave up
implementing multiple Paxos state machine in the same node) so each replica's
tablet holds multiple directories. Since there are many server nodes in a
typical deployment, the placement driver and the zonemasters will be able to
group server nodes to form Paxos groups of size 3, 5 or more depending on
application configurations.

\subsection{Considerations for Two-Phase Commit}
The two-phase commit has been regarded as ``anti-availability'' due to its
various shortcomings in terms of availability requirements for both
participants, coordinator, and even the clients~\cite{helland2007life}.
Some storage service tries to code around it and provide a weaker isolation
level (if it supports some form of transactions) and weaker consistency
level for the sake of better availability~\cite{gilbert2002brewer}.

Google's work on Spanner shows a promising result: when building two-phase
commit on a much higher level of availability unit: a Paxos group, two-phase
commit not only becomes usable, but also allows us to harness the good old
transaction processing studies on isolation: two-phase locking and well-ordered locking enables {\Sconsistent}~\cite{gray1992transaction}.


\section{Concurrency Control}
The techniques for ensuring {\Sconsistent} is well-discussed in~\cite{gray1992transaction}.
It is the problem of {\Lconsistent} and how it is solved makes Spanner so
interesting. {\Lconsistent} requires system reads and
writes respects ``real-time'' ordering, in that the reads should reflect the
writes happened before them, and the writes should be applied in the order of
their commit timestamps.

At a high level, Spanner commit protocol still is a variant of two-phase
commit. To ensure external consistency, Spanner timestamps every
mutation (writes) of the database, and make sure that these mutations
are available for subsequent reads/writes only after the absolute real-time
has passed those mutations' commit timestamps.

To serve reads, a read-only transaction $T$ will carry a timestamp $t_{
read}$ for locating the point in time of the reading, so that Spanner can
ensure the data is no earlier than this timestamp. When $t_{read} \ge
t_{abs}(T_{commit})$, the reads will always reflect the latest writes.

Roughly speaking, the modification of the two-phase commit done in Spanner is
for negotiating a lower bound for the timestamps of each committing
transaction. The negotiated timestamp is then used by each replica to keep
track if it is safe to serve a read when there are in-flight transactions.
Also, after two-phase commit is completed, the latest committed mutation will
be used to keep track if this replica has up-to-date states for reading.

\subsection{Paxos Group States}
The smallest unit for participating two-phase commit is a Paxos group. Each
group has a leader, and it also maintains the lock table and works as the
participating transaction manager for the group's other replicas.

The Paxos group maintains a replicated log, which is used to persisting commit
or prepares records. When a transaction is committed, sometimes later, the replicas
of the group is allowed to tail this log and apply the commit message onto its
local persisted state machine. In the case of Spanner, they use a $B^+$-tree
as both the index structure and the data storage for this state machine.

Each replica in the group will have the following internal states in addition
to the Paxos log and the state machine:
\begin{itemize}
   \item $t_{safe}^{TM}$
      This timestamp indicates a strict lower-bound for all in-flight
      transactions' commit timestamp. The meaning of this timestamp is, if
      there is a read translation at time $s_{read}$ and $s_{read} \le
      t_{safe}^{TM}$, then the in-flight transactions cannot interfere this
      read as the mutation will happen in a future timestamp with respect to
      $s_{read}$. The transaction protocol entails how this state is maintained
      for each committed transition.
   \item $t_{safe}^{Paxos}$
      This timestamp indicates a lower-bound for the next future transaction's
      commit timestamp. In other words, the next mutation's commit point cannot
      be earlier than $t_{safe}^{Paxos}$. The meaning of this timestamp is,
      if there is a read translation at time $s_{read}$ and $s_{read} \le
      t_{safe}^{Paxos}$, then the replica can safely serve a read by reading
      the persisted state machine, since the replica knows the next mutation
      will happen in a future time with respect to $s_{read}$.
   \item $t_{latest}$
      This timestamp indicates the latest committed record in the Paxos log.
      One may argue that this is redundant since we can trivially have
      $t_{safe}^{Paxos} = t_{latest}$. In fact, Spanner makes it
      $t_{safe}^{Paxos} \ge t_{latest}$, for the reasons to be entailed later.
   \item $n_{latest}$
      This is a sequence number representing the latest commit record in the
      Paxos log.
   \item $MinNextTS(n)$
      This maps a sequence number to a timestamp. $MinNextTS(n)$ gives a
      lower-bound for the committing timestamp this replica will propose for
      commit record $n+1$. This is used to compute $t_{safe}^{Paxos}$.
\end{itemize}

\subsection{Committing Read-Write Transactions}
the transaction involves both read and write are declared by the client and
marked as $rw$ transactions. The rough transaction $T_i$'s committing process
is like this:
\begin{itemize}
   \item The client identifies all the groups involved in $T_i$, and pick one
      group $g_{c}$ as the coordinator, then the rest of the groups $g_{j}$
      become the participants for this two-phase commit. From now on, each
      group is identified by its Paxos leader node. i.e. $g$ is the leader node
      for that group, not all the replicas in the group.
   \item The client sends the commit message to $g_{c}$, and all buffered
      writes to each corresponding group. Since different groups are usually in
      different zones connected by WAN, let the client drive the commit process
      avoids sending these buffered data twice over the WAN.
   \item The non-coordinator groups $g_{j}$ will then start preparing for $T_i$.
      It first acquires the locks in the lock table, then proposes a committing
      timestamp to the coordinator. This timestamp is called ``prepare timestamp''
      $s_{i, g}^{prepare}$
      and will be logged as a prepare message in its Paxos log. $s_{i, g}^{prepare}$
      has to satisfy:
      \begin{equation}
         s_{i, g}^{prepare} \ge t_{safe, g}^{Paxos}
      \end{equation}
      So that when the transaction commits, $t_{safe, g}^{Paxos}$ serves its
      purpose.
   \item The coordinating group $g_c$ will also acquires locks in the group
      for $T_i$. After it receives all the prepare messages, it will pick
      a ``committing timestamp'' $s_i$ such that
      \begin{equation}
         \begin{split}
         \forall g, s_i \ge s_{i, g}^{prepare} \\
         s_i \ge TT.now().latest \\
         s_i \ge t_{latest}
         \end{split}
      \end{equation}
      Then the transaction is now committed, the coordinator writes a commit
      message to its own Paxos log.
   \item Here is the real trick: after coordinator declares commit for $T_i$,
      it does not immediately make it visible to the rest of the participant
      group \textit{nor should it allow for the locks to be released}.
      Instead, it simply waits. While the rest of the participants are
      waiting for the coordinator to declare commit, the coordinator just
      wait until $TT.after(s_i)$ is true. It is only after this is true, the
      coordinator can broadcast the commit message to the rest of the
      participants, and they can apply this commit message to their Paxos
      log, and subsequently, their state machines, to make this commit
      visible and release the locks. This imposes the real-time ordering for
      subsequent read or write transactions. To wit, consider another
      transaction $T_j$ issued by the client such that:
      \[
         t_{abs}(T_j^{start}) \ge t_{abs}(T_i^{commit})
      \]
      and, by definition, the real time commit point $t_{abs}(T_i^{commit})$
      is when $T_i$ is visible, which satisfies:
      \[
         s_i \le t_{abs}(T_i^{commit})
      \]
      Thanks to the commit wait, and we can thus conclude:
      \[
         s_j \ge t_{abs}(T_j^{start}) \ge t_{abs}(T_i^{commit}) \ge s_i
      \]
      In other words, the internal ordering of $T_i$ and $T_j$ mutating the
      state machine respects the real time ordering.

\end{itemize}

\subsection{Read-Only Transactions}
To serve read, read transaction will carry a timestamp $s_{read}$ to specify
how up-to-date the read should be served. Setting $s_{read} = TT.now.latest$
will ensure {\Lconsistent}. It is now a matter of how each replica keeps
track of how up-to-date it is when serving the read at $s_{read}$.
There are nature problem arises:
\begin{itemize}
   \item Each replica for the same group will apply commit record at different
      speed to the persistent layer, and thus is not up-to-date with respect to each other within
      the group.
   \item When a read-only transaction expands multiple groups, only all the
      groups are up-to-date can they serve the read at $s_{read}$.
   \item When each replica is handling some in-flight transaction (i.e. a
      prepare message is present), the replica needs to know if this
      transaction, after it is committed, can interfere with the read at
      $s_{read}$. If the in-flight transaction has a timestamp $s \le
      s_{read}$, then this replica cannot serve read until the transaction
      commits.
   \item Finally, if the replica knows its last committed transaction timestamp
      $t_{latest} \ge s_{read}$, then its state is up-to-date, but if
      $t_{latest} < s_{read}$, can it serve the read? Once the system stops
      writing and only see read requests, eventually $t_{latest} < s_{read}$.
      How can the replica tell if it is OK to serve read?
\end{itemize}

To answer these questions, Spanner uses two states $t_{safe}^{TM}$ and
$t_{safe}^{Paxos}$ to help determine if read at $s_{read}$ can be safely served
or not. The high-level idea is to track committed and in-flight transaction
timestamps, and derive a lower-bound timestamp for both of them. If the read
timestamp is no later than the lower-bound, then the read is safe to serve.

$t_{safe}^{TM}$ is derived in the following manner: the replica tracks all the
prepare transactions $T_i$ it current sees without commit. If there is no such
$T_i$, then $t_{safe}^{TM} = \infty$. Otherwise, recall that each replica will
see the prepare message in the Paxos group. For each such message uncommitted,
$t_{g, i}^{prepare} \le t_{g, i}^{commit}$ guaranteed by the read-write
transaction commit protocol. Then it is suffice to set $t_{safe}^{TM}
=\min_i\{t_{g, i}^{prepare}\} - 1$.

$t_{safe}^{Paxos}$ is considerably trickier. It is tempting for each replica to
set $t_{safe}^{Paxos} = t_{latest}$, since timestamp marches forward trivially.
But this cannot work since when there is no more write transaction, future
reads cannot be served. Even for a read that happens before some latest write,
the read can span to multiple groups and some groups did not participate that
write transaction. A more practical way to look at $t_{safe}^{Paxos}$ is to
regard it as the lower-bound for the next write operation (at sequence number
$n+1$ provided the current latest applied commit record is $n$). As long as
$s_{read}$ is before the future write timestamp, no mutation can happen
between $s_{read}$ and the future write. Thus the read can be served. Recall
that each group honors its current $t_{safe}^{Paxos}$ when proposing a commit
timestamp for the prepare message, so the next write timestamp can be
enforced. The leader simply keeps a mapping $MinNextTS(n)$ that maps $n$ to
the next write $n+1$'s timestamp lower-bound, and uses it to advance
$t_{safe}^{Paxos}$ in the group. Such a mapping is done by extending
$MinNextTS(n)$ every 8 seconds. i.e. The next proposed commit timestamp will
be at most 8 seconds ahead from $TT.now.latest$. During this 8 second period,
the replica can safely serve reads with $s_{read} = TT.now.latest$.

But how long should we wait? Does that mean the write transaction needs to
wait for 8 seconds in the worst case by the coordinator? A leader should not
use $MinNextTS(n)$ blindly, since this enforces a lower-bound for commit
timestamp (and the coordinator needs to wait for that to be passed). Only
when a read occurred and $t_{safe}^{Paxos}$ is out of sync with respect to
current $TT.now$ too far, the leader can use $MinNextTS(n)$ to advance
$t_{safe}^{Paxos}$ to serve the read (but the writes in this
timeframe are poisoned by at most 8 seconds).

Also, non-conflict reads can be arbitrarily delayed by the (lack of)
advancement of the above two states. For example, prepared yet not committed
transition timestamps gives a false conflict if their timestamps are too
early. Spanner's answer is to use a more fine-grained table for the key
ranges' $t_{safe}^{TM}$ and $t_{safe}^{Paxos}$. So that when the read
transaction comes, non-conflict ranges' timestamps do not contribute to the
comparison, only the overlapping ranges' matter matters.

\bibliography{main}{}
\bibliographystyle{plain}
\end{document}
