\documentclass[12pt]{article}
\usepackage{amsmath}
\usepackage{euler}
\usepackage{authblk}
\usepackage{tikz}
\usetikzlibrary{positioning}

\usepackage{hyperref}
\hypersetup{
    colorlinks,
    citecolor=black,
    filecolor=black,
    linkcolor=black,
    urlcolor=black
}

\usepackage{bookmark}
\usepackage{cite}
\usepackage{fancyhdr}

\usepackage[margin=1.13in]{geometry}

\pagestyle{fancy}
\fancyhf{}
\rhead{\rightmark}

\usepackage{titlesec}
\titleformat{\chapter}[display]
  {\normalfont\bfseries}{}{0pt}{\Huge}
\author{Junlong Gao}
\affil{}
\date{}

\newcommand{\Lconsistent}{linearizability }
\newcommand{\Sconsistent}{serializability }
\newcommand{\LConsistent}{Linearizability }
\newcommand{\SConsistent}{Serializability }

\newcommand{\wip}[1]{\textcolor{red}{[\textit{TODO: #1}]}}

\begin{document}
Append-only, randomly collected thoughts, remarks, notes, formulas, and
a right amount of mathematics.

\begin{flushright}
    \textit{
Explain to me as if this is five months after.
    }
\end{flushright}

\notesep
\subsection*{MacVlan}
Here is a good overview on what is
\href{https://hicu.be/bridge-vs-macvlan}{Linux MacVlan} and how docker uses
its bridge mode.

MacVlan is a software feature in Linux kernel to attach virtual mac addresses
to a physical layer 2 device. The new mac addresses are said to be the
``child'' of the parent network device, and can be used to configure slave
(usually virtual) devices used in containers or VMs. MacVlan can run in
multiple modes, the one supported in docker is the ``bridge''. This allows
different child mac addresses to be able to routed to each other when they
arrive at the parent device. This effectively turns the parent device into a
switch (``bridge''), hence having the name.

Following
\href{https://hicu.be/docker-networking-macvlan-vlan-configuration}{this},
one can setup containers using macvlan. Each container will be associated
with a virtual interface with mac address, hence able to get an ip by the
docker runtime. On the other hand, docker internal IP implementation does not
allow for DHCP. On the host running containers, container can ping each other
as they are using the bridge mode of the same physical interface macvlan, but
cannot ping the host. To ping host, one need to create a link using an
additional macvlan and give it an IP. This is the internal constraint in
macvlan: child mac cannot be routed to the parent mac in that same network
interface. In fact, it is filtered out in the first hop. As an interesting
consequence, outside of the host, I can ping both the host and the container
via the ip associated with that interface. So each mac will have an IP (ARP
protocol), but one interface can have multiple mac address hence multiple ip
via macvlan.

\subsection*{Linux Programming Interfaces Today}
Traditional operating system course in a computer science class will introduce
file system as a storage abstraction (file as storage access gateways), sockets
and DNS as network abstractions (naming and transportation), and thread/process
as a compute abstraction (sharing, CPU, memory, etc).

Virtualization technologies was a first step to cope with the increasing
compute, storage and networking hardwares. Applications, however monolithic
they tend to be, are mostly unable to saturate a single hosts' power. As a by
product, they achieve better isolation and provide opportunities for better
management.

Containerization technologies were almost as early as when virtual machines
began the fun, but since they are highly operating system dependent (Google
was among the first a few hacking and upstreaming the patches for \texttt{cgroup}
and \texttt{namespaces}), they did not attract so much attention nor having a
commercial success like other virtual machine vendors.

It seems like today due to the domination of the Linux kernel and huge
success in the open source ecosystem, most of the applications backing the
Internet are on Linux. This fuels the new programming abstractions people use
to build applications. Knowing the mature support in containerization, as
well as the requirements for scaling applications for Internet load,
applications are built and deployed not in an operating system like they
would be when running inside a virtual machine. Instead, they become
containers and assumes little about the underlying storage nor networking.

Networking are getting more and more complicated in the world of containers.
Applications are not only requiring an IP address to talk to each other or
consume an API. Networks are further divided via the network namespaces like
the Pod Abstraction in Kubernetes, traffics are usually load-balanced and
routed to different backend services, and tracing, logging, health checking
and service discoveries are ubiquitous in today's backend design. These are
far beyond what a socket can offer. Tools like
\href{https://www.envoyproxy.io/}{Envoy} are built to create this ``Service
Mesh'' abstraction and provide these services, as well as some existing tools
like
\href{https://www.nginx.com/resources/glossary/layer-7-load-balancing/}{Nginx}
and application libraries.

What abstractions should storage systems provide? Local file system was how
everything begins, then follows richer semantics like Key-Value and SQL.
Access path to the storage layer is no longer a simple local syscall. Network
hops, placement decisions and data sharding are all new considerations in a
storage service. Open file for writing, list directories and files, these are
no longer what applications need. Structured data are stored in a relational
manner, and unstructured, large, streaming data are either in a blob/object
store, or piped back and forth via a streaming service.

\notesep
\end{document}