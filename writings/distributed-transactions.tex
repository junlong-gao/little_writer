\documentclass[12pt]{article}
\usepackage{amsmath}
\usepackage{euler}
\usepackage{authblk}
\usepackage{tikz}
\usetikzlibrary{positioning}

\usepackage{hyperref}
\hypersetup{
    colorlinks,
    citecolor=black,
    filecolor=black,
    linkcolor=black,
    urlcolor=black
}

\usepackage{bookmark}
\usepackage{cite}
\usepackage{fancyhdr}

\usepackage[margin=1.13in]{geometry}

\pagestyle{fancy}
\fancyhf{}
\rhead{\rightmark}

\usepackage{titlesec}
\titleformat{\chapter}[display]
  {\normalfont\bfseries}{}{0pt}{\Huge}
\author{Junlong Gao}
\affil{}
\date{}

\newcommand{\Lconsistent}{linearizability }
\newcommand{\Sconsistent}{serializability }
\newcommand{\LConsistent}{Linearizability }
\newcommand{\SConsistent}{Serializability }

\newcommand{\wip}[1]{\textcolor{red}{[\textit{TODO: #1}]}}

\begin{document}
\tableofcontents

\title{Distributed Transactions}
\maketitle

Atomicity, Consistency, Isolation Levels, and Durability (ACID), are the
corner stones for the storage system programing interface:
transactions~\cite{gray1992transaction}. In reality, these promises are
harder and harder to keep when increasing the scale of the storage
system~\cite{bailis2013highly}.

Why distributed transactions? We need distribution in storage systems for
mainly two purposes: scalability of performances and redundancy for
reliability. In some systems, redundancy requires distributed transactions to
ensure consistency across replicas, this includes some virtual SAN
implementation and file systems~\cite{ghemawat2003google}. The form of
transactions in replication is somewhat simpler: their main purpose is for
Durability and Atomicity. 

In some other recent architectures, where storage system is implemented with
two layers\cite{Cooper_2013, cao2018polarfs,weil2006ceph,
verbitski2017amazon, peng2010large}, replication is pushed down to the lower
layer and can use other approaches like replicated logs or gossip for
replication. In these cases, when the entire system storage is sharded for
scalability, transactions involves multiple resources manager will require
some form of two-phase commit to implement ACID semantics.

But the boundary of these use-cases is not black-and-white. Imagine
implementing a virtual SAN system with RAID-6 erasure coding for data
blocks. There are 4 data nodes and 2 syndrome nodes, and redundancy is
achieved by contacting any 4 nodes out of the 6 to recover a data
block~\cite{plank2013erasure}. In this case, is it replication, or sharding?
Erasure coding works by generating 6 syndrome blocks from the 4 data blocks
(one coding group), and two-phase commit is used to ensure these 6 pieces of
data end up all-or-nothing durably on the 6 nodes. On the other handing,
reading the 4 consecutive blocks in the same coding group can be distributed
to any 4 of the 6 nodes, so read performance benefits from this distribution.
There is no obvious replicated state machine approach the author know of to
achieve this, hence two-phase commit is used.

The discussion gets even further into what kind of programming interface
these distributed transactions offers. Is it key-value space, or Structured
Query Language? In traditional MySQL implementation, SQL is done on a b-plus
tree, which is effectively a key-value storage engine. Now it is interesting
to see that when implementing a distributed SQL database, it is usually done
over a thin SQL layer on a distributed key-value storage layer. There is
a rabbit hole on how journaling should be done in key-value layer and b-plus
tree layer~\cite{gray1992transaction, sippu2015transaction}.

This article explores the distributed transactions and architectures entails
in the two-layered approaches.

\section{Physical and Logical Databases}
In traditional centralized database designs, the system is already in 2
layers internally. SQL operations always generate key-value operations on the
b-plus tree, and the b-plus tree is the physical representation of these
tuples. The physical database, sometimes called ``the storage engine'', is
physical in a sense that they are closer to disk (b-plus tree pages are
literally disk pages), and relies on the underlying hardware for page
atomicity (but no structured update). The logical database relies on the
storage engine for storing the tuple, yet itself can have its own logging and
in-memory state like cache. In traditional design, mutation of the database
is expressed in the logical layer by translating SQL to key-value operations, and
then translated to the physical structure mutation, and journaled for
atomicity and durability.

For Google Spanner and Amazon Aurora, this relationship is turned
``inside-out''~\cite{Cooper_2013, verbitski2017amazon}, in the sense that
logical mutations are first-class citizens in the data path instead of hidden
beneath the internal structure. In fact, some open-source implementation like
TiDB explicitly allows using the underlying logical database TiKV directly.
In these cases, mutations must first be done in a logical manner:
key-value operations are logged in the persistent storage, and picked up by
the physical layer to replay onto the physical database, regardless which
storage engine it uses. Popular implementations of LSM tree shows its value
in write-intensive workloads.

\section{Jounraling}
The two-layered approach can benefit from the underlying key-value store
being both highly available and highly scalable. Logical journaling allows
this separation so that one can focus on solving data reliability and
availability problems in the key-value layer. Also, journaling at the logical
layer reduces write-amplification dramatically over the
network~\cite{verbitski2017amazon}. Finally, logical journaling is also
naturally suitable for consensus protocols implementing replicated logs like
Raft, provided records in the logical journal are idempotent. In AWS Aurora,
they have their own implementation of replicated log based on gossip and
checkpoint progressing~\cite{verbitski2018amazon} (curiously, there is no
notion of distributed transactions as there is no sharding in Aurora either).

Now it is worth mention the differences in Aurora and the rest of the world.
Aurora is using a more involved form called Physiological Logging as this is
what original MySQL uses~\cite{sippu2015transaction}. The log records
describes how the b-plus tree pages to be mutated in terms of logical
key-value operations, without providing the entire post-image of the page. In
this sense, the write path will still have to stick with the traditional
database design where latches are used to guard pages dirtied without
sharding to different nodes, and in this case Aurora can only support one
writer. In this sense, the two layers in Aurora between the logical layer and
the physical layer are still interleaved.

In Spanner however, transactions are done in the key-value and key-value form
only, and there is no dependency on the underlying physical database. Writing
is  a matter of logging the key-value pairs in the Paxos log, and no need to
handle dirty page latches in a distributed manner. Then, sharding can be done
in the key space and each shard has a Paxos file for its share of journaling.
To make sure the update across shards are updated atomically and durability,
an additional two-phase commit protocol is thus required.

\section{Timestamps and Concurrency}
\wip{Explain serailibility and linearizability and how timestamps can be used}

\bibliography{main}{}
\bibliographystyle{plain}
\end{document}
